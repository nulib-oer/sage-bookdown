\documentclass[openany]{book}
\usepackage{lmodern}
\usepackage{amssymb,amsmath}
\usepackage{ifxetex,ifluatex}
\usepackage{fixltx2e} % provides \textsubscript
\ifnum 0\ifxetex 1\fi\ifluatex 1\fi=0 % if pdftex
  \usepackage[T1]{fontenc}
  \usepackage[utf8]{inputenc}
\else % if luatex or xelatex
  \ifxetex
    \usepackage{mathspec}
  \else
    \usepackage{fontspec}
  \fi
  \defaultfontfeatures{Ligatures=TeX,Scale=MatchLowercase}
\fi
% use upquote if available, for straight quotes in verbatim environments
\IfFileExists{upquote.sty}{\usepackage{upquote}}{}
% use microtype if available
\IfFileExists{microtype.sty}{%
\usepackage[]{microtype}
\UseMicrotypeSet[protrusion]{basicmath} % disable protrusion for tt fonts
}{}
\PassOptionsToPackage{hyphens}{url} % url is loaded by hyperref
\usepackage[unicode=true]{hyperref}
\hypersetup{
            pdftitle={Bookdown with Sage Cells},
            pdfauthor={Author Name},
            pdfborder={0 0 0},
            breaklinks=true}
\urlstyle{same}  % don't use monospace font for urls
\usepackage{natbib}
\bibliographystyle{plainnat}
\usepackage{longtable,booktabs}
% Fix footnotes in tables (requires footnote package)
\IfFileExists{footnote.sty}{\usepackage{footnote}\makesavenoteenv{long table}}{}
\usepackage{graphicx,grffile}
\makeatletter
\def\maxwidth{\ifdim\Gin@nat@width>\linewidth\linewidth\else\Gin@nat@width\fi}
\def\maxheight{\ifdim\Gin@nat@height>\textheight\textheight\else\Gin@nat@height\fi}
\makeatother
% Scale images if necessary, so that they will not overflow the page
% margins by default, and it is still possible to overwrite the defaults
% using explicit options in \includegraphics[width, height, ...]{}
\setkeys{Gin}{width=\maxwidth,height=\maxheight,keepaspectratio}
\IfFileExists{parskip.sty}{%
\usepackage{parskip}
}{% else
\setlength{\parindent}{0pt}
\setlength{\parskip}{6pt plus 2pt minus 1pt}
}
\setlength{\emergencystretch}{3em}  % prevent overfull lines
\providecommand{\tightlist}{%
  \setlength{\itemsep}{0pt}\setlength{\parskip}{0pt}}
\setcounter{secnumdepth}{5}
% Redefines (sub)paragraphs to behave more like sections
\ifx\paragraph\undefined\else
\let\oldparagraph\paragraph
\renewcommand{\paragraph}[1]{\oldparagraph{#1}\mbox{}}
\fi
\ifx\subparagraph\undefined\else
\let\oldsubparagraph\subparagraph
\renewcommand{\subparagraph}[1]{\oldsubparagraph{#1}\mbox{}}
\fi

% set default figure placement to htbp
\makeatletter
\def\fps@figure{htbp}
\makeatother

\usepackage{booktabs}
\usepackage{amsthm}
\usepackage{environ}
\makeatletter
\def\thm@space@setup{%
  \thm@preskip=8pt plus 2pt minus 4pt
  \thm@postskip=\thm@preskip
}
\makeatother
\NewEnviron{compute}{}{}

\title{Bookdown with Sage Cells}
\author{Author Name}
\date{7/6/2020}

\begin{document}
\maketitle

{
\setcounter{tocdepth}{1}
\tableofcontents
}
\chapter*{Preface}\label{preface}
\addcontentsline{toc}{chapter}{Preface}

This is an example of a \href{https://bookdown.org}{Bookdown} project
with \href{https://sagecell.sagemath.org}{SageMathCell}. Sample content
is provided by \href{http://linear.ups.edu/html/fcla.html}{Beezer, R.
(2015). A first course in linear algebra} under the terms of the GNU
Free Documentation License.

\begin{compute}
\end{compute}

\chapter{Reduced Row-Echelon Form}\label{reduced-row-echelon-form}

\textbf{By Robert Beezer}
\emph{\href{https://github.com/rbeezer/fcla/blob/master/src2/section-RREF.xml}{Here
is the PreTeXt version of the same chapter}}

After solving a few systems of equations, you will recognize that it
does not matter so much what we call our variables, as opposed to what
numbers act as their coefficients. A system in the variables
\(x_1,\,x_2,\,x_3\) would behave the same if we changed the names of the
variables to \(a,\,b,\,c\) and kept all the constants the same and in
the same places. In this section, we will isolate the key bits of
information about a system of equations into something called a matrix,
and then use this matrix to systematically solve the equations. Along
the way we will obtain one of our most important and useful
computational tools.

\section{Matrix and Vector Notation for Systems of
Equations}\label{matrix-and-vector-notation-for-systems-of-equations}

\textbf{Definition M: Matrix} An \(m\times n\) is a rectangular layout
of numbers from \(complexes\) having \(m\) rows and \(n\) columns. We
will use upper-case Latin letters from the start of the alphabet
\(A,\,B,\,C,\dotsc\) to denote matrices and squared-off brackets to
delimit the layout. Many use large parentheses instead of brackets ---
the distinction is not important. Rows of a matrix will be referenced
starting at the top and working down (i.e.~row 1 is at the top) and
columns will be referenced starting from the left (i.e.~column 1 is at
the left). For a matrix \([A]\), the notation \([{A}{ij}]\) will refer
to the complex number in row \(i\) and column \(j\) of \(A\).

Be careful with this notation for individual entries, since it is easy
to think that \([{A}{ij}]\) refers to the \emph{whole} matrix. It does
not. It is just a \emph{number}, but is a convenient way to talk about
the individual entries simultaneously. This notation will get a heavy
workout once we get to
\href{http://linear.ups.edu/html/chapter-M.html}{Chapter M}.

\textbf{Example AM: A Matrix}

\[B=\begin{bmatrix}
-1&2&5&3\\
1&0&-6&1\\
-4&2&2&-2
\end{bmatrix}\]

is a matrix with \(m=3\) rows and \(n=4\) columns. We can say that
\([{B}{2,3}=-6]\) while \([{B}{3,4}=-2]\).

\textbf{Sage M: Matrices}

Matrices are fundamental objects in linear algebra and in Sage, so there
are a variety of ways to construct a matrix in Sage. Generally, you need
to specify what types of entries the matrix contains (more on that in a
minute), the number of rows and columns, and the entries themselves.
First, let us dissect an example:

\begin{compute}
\end{compute}

QQ is the set of all rational numbers (fractions with an integer
numerator and denominator), 2 is the number of rows, 3 is the number of
columns. Sage understands a list of items as delimited by brackets
({[},{]}) and the items in the list can again be lists themselves. So
{[}{[}1, 2, 3{]}, {[}4, 5, 6{]}{]} is a list of lists, and in this
context the inner lists are rows of the matrix.

There are various shortcuts you can employ when creating a matrix. For
example, Sage is able to infer the size of the matrix from the lists of
entries.

\begin{compute}
\end{compute}

Or you can specify how many rows the matrix will have and provide one
big grand list of entries, which will get chopped up, row by row, if you
prefer.

\begin{compute}
\end{compute}

It is possible to also skip specifying the type of numbers used for
entries of a matrix, however this is fraught with peril, as Sage will
make an informed guess about your intent. Is this what you want?
Consider when you enter the single character ``2'' into a computer
program like Sage. Is this the integer \(2\), the rational number
\(\frac{2}{1}\), the real number \(2.00000\), the complex number
\(2 + 0i\), or the polynomial \(p(x)=2\)? In context, us humans can
usually figure it out, but a literal-minded computer is not so smart. It
happens that the operations we can perform, and how they behave, are
influenced by the type of the entries in a matrix. So it is important to
get this right and our advice is to be explicit and be in the habit of
always specifying the type of the entries of a matrix you create.

Mathematical objects in Sage often come from sets of similar objects.
This set is called the ``parent'' of the element. We can use this to
learn how Sage deduces the type of entries in a matrix. Execute the
following three compute cells in the Sage notebook, and notice how the
three matrices are constructed to have entries from the integers, the
rationals and the reals.

\begin{compute}
\end{compute}

\begin{compute}
\end{compute}

\begin{compute}
\end{compute}

Sage knows a wide variety of sets of numbers. These are known as
``rings'' or ``fields'' (see Section F), but we will call them ``number
systems'' here. Examples include: \texttt{ZZ} is the integers,
\texttt{QQ} is the rationals, \texttt{RR} is the real numbers with
reasonable precision, and \texttt{CC} is the complex numbers with
reasonable precision. We will present the theory of linear algebra over
the complex numbers. However, in any computer system, there will always
be complications surrounding the inability of digital arithmetic to
accurately represent all complex numbers. In contrast, Sage can
represent rational numbers exactly as the quotient of two (perhaps very
large) integers. So our Sage examples will begin by using QQ as our
number system and we can concentrate on understanding the key concepts.

Once we have constructed a matrix, we can learn a lot about it (such as
its parent). Sage is largely object-oriented, which means many commands
apply to an object by using the ``dot'' notation. \texttt{A.parent()} is
an example of this syntax, while the constructor
\texttt{matrix({[}{[}1,\ 2,\ 3{]},\ {[}4,\ 5,\ 6{]}{]})} is an
exception. Here are a few examples, followed by some explanation:

\begin{compute}
\end{compute}

\begin{compute}
\end{compute}

\begin{compute}
\end{compute}

\begin{compute}
\end{compute}

The number of rows and the number of columns should be apparent,
\texttt{.base\_ring()} gives the number system for the entries, as
included in the information provided by \texttt{.parent()}.

Computer scientists and computer languages prefer to begin counting from
zero, while mathematicians and written mathematics prefer to begin
counting at one. Sage and this text are no exception. It takes some
getting used to, but the reasons for counting from zero in computer
programs soon becomes very obvious. Counting from one in mathematics is
historical, and unlikely to change anytime soon. So above, the two rows
of A are numbered 0 and 1, while the columns are numbered 0, 1 and 2. So
\texttt{A{[}1,2{]}} refers to the entry of \texttt{A} in the second row
and the third column, i.e. \texttt{6}

There is much more to say about how Sage works with matrices, but this
is already a lot to digest. Use the space below to create some matrices
(different ways) and examine them and their properties (size, entries,
number system, parent).

\begin{compute}
\end{compute}

When we do equation operations on system of equations, the names of the
variables really are not very important. Use \(x_1\), \(x_2\), \(x_3\),
or \(a\), \(b\), \(c\), or \(x\), \(y\), \(z\), it really does not
matter. In this subsection we will describe some notation that will make
it easier to describe linear systems, solve the systems and describe the
solution sets. Here is a list of definitions, laden with notation.

\textbf{Definition CV: Column Vector.} A \emph{column vector} of
\emph{size} \(m\) is an ordered list of \(m\) numbers, which is written
in order vertically, starting at the top and proceeding to the bottom.
At times, we will refer to a column vector as simply a \emph{vector}.
Column vectors will be written in bold, usually with lower case Latin
letter from the end of the alphabet such as \$\{u\}, \$\{v\}, \$\{w\},
\$\{x\}, \({y}\), \({z}\). Some books like to write vectors with arrows,
such as \({u}\). Writing by hand, some like to put arrows on top of the
symbol, or a tilde underneath the symbol, as in
\(\underset{\sim}{\textstyle u}\). To refer to the \emph{entry} or
\emph{component} of vector \({v}\) in location \(i\) of the list, we
write \([{v}]_{i}\).

Be careful with this notation. While the symbols \({v}_{i}\) might look
somewhat substantial, as an object this represents just one entry of a
vector, which is just a single complex number.

\textbf{Definition ZCV Zero Column Vector} The zero vector of size m is
the column vector of size m where each entry is the number zero,

\[
0 = \begin{bmatrix}
     0 \\
     0 \\
     0 \\
     \vdots \\
     0 \\
     \end{bmatrix}
\]

or defined much more compactly, \([0]_{i}\) for \(1\leq i\leq m\).

\protect\hyperlink{definition-cm-coefficient-matrix}{\textbf{Coefficient
Matrix}}

For a system of linear equations,

\begin{align*}
a_{11}x_1+a_{12}x_2+a_{13}x_3+\dots+a_{1n}x_n&=b_1\\
a_{21}x_1+a_{22}x_2+a_{23}x_3+\dots+a_{2n}x_n&=b_2\\
a_{31}x_1+a_{32}x_2+a_{33}x_3+\dots+a_{3n}x_n&=b_3\\
\vdots&\\
a_{m1}x_1+a_{m2}x_2+a_{m3}x_3+\dots+a_{mn}x_n&=b_m
\end{align*}

the \emph{coefficient matrix} is the \$m\textbackslash{}times n\$ matrix

\[
A=
\begin{bmatrix}
a_{11}&a_{12}&a_{13}&\dots&a_{1n}\\
a_{21}&a_{22}&a_{23}&\dots&a_{2n}\\
a_{31}&a_{32}&a_{33}&\dots&a_{3n}\\
\vdots&\\
a_{m1}&a_{m2}&a_{m3}&\dots&a_{mn}\\
\end{bmatrix}
\]

\protect\hyperlink{definition-voc-vector-of-constants}{\textbf{Vector of
Constants}}

For a system of linear equations,

\[\begin{aligned}
a_{11}x_1+a_{12}x_2+a_{13}x_3+\dots+a_{1n}x_n&=b_1\\
a_{21}x_1+a_{22}x_2+a_{23}x_3+\dots+a_{2n}x_n&=b_2\\
a_{31}x_1+a_{32}x_2+a_{33}x_3+\dots+a_{3n}x_n&=b_3\\
\vdots&\\
a_{m1}x_1+a_{m2}x_2+a_{m3}x_3+\dots+a_{mn}x_n&=b_m
\end{aligned}\]

\protect\hyperlink{definition-solv-solution-vector}{\textbf{Solution
Vector}}

For a system of linear equations,

\[\begin{aligned}
a_{11}x_1+a_{12}x_2+a_{13}x_3+\dots+a_{1n}x_n&=b_1\\
a_{21}x_1+a_{22}x_2+a_{23}x_3+\dots+a_{2n}x_n&=b_2\\
a_{31}x_1+a_{32}x_2+a_{33}x_3+\dots+a_{3n}x_n&=b_3\\
\vdots&\\
a_{m1}x_1+a_{m2}x_2+a_{m3}x_3+\dots+a_{mn}x_n&=b_m
\end{aligned}\]

The solution vector may do double-duty on occasion. It might refer to a
list of variable quantities at one point, and subsequently refer to
values of those variables that actually form a particular solution to
that system.

\protect\hyperlink{definition-mrls-matrix-representation-of-a-linear-system}{\textbf{Matrix
Representation of a Linear System}}. If \$A\$ is the coefficient matrix
of a system of linear equations and \({b}\) is the vector of constants,
then we will write \(\$\mathcal L \mathcal S({A}, b)\) as a shorthand
expression for the system of linear equations, which we will refer to as
the \emph{matrix representation} of the linear system.

\protect\hyperlink{example-nsle-notation-for-systems-of-linear-equations}{\textbf{Notation
for systems of linear equations}}

The system of linear equations

\[\begin{aligned}
2x_1+4x_2-3x_3+5x_4+x_5&=9\\
3x_1+x_2+\quad\quad x_4-3x_5&=0\\
-2x_1+7x_2-5x_3+2x_4+2x_5&=-3
\end{aligned}\]

has coeffient matrix

\[
A=
\begin{bmatrix}
2 & 4 & -3 & 5 & 1\\
3 & 1 & 0 & 1 & -3\\
-2 & 7 & -5 & 2 & 2
\end{bmatrix}
\]

and vector of constants

\[
{b}=\begin{bmatrix}
9\\
0\\
-3
\end{bmatrix}
\]

and so will be referenced as \(\$\mathcal L \mathcal S({A}, b)\)

\protect\hyperlink{definition-am-augmented-matrix}{\textbf{Augmented
Matrix}}. Suppose we have a system of \$m\$ equations in \$n\$
variables, with coefficient matrix \$A\$ and vector of constants
\$\textbackslash{}vect\{b\}\$. Then the \emph{augmented matrix} of the
system of equations is the \$m\textbackslash{}times(n+1)\$ matrix whose
first \$n\$ columns are the columns of \$A\$ and whose last column
(\$n+1\$) is the column vector \$\textbackslash{}vect\{b\}\$. This
matrix will be written as
\$\textbackslash{}augmented\{A\}\{\textbackslash{}vect\{b\}\}\$.

The augmented matrix \emph{represents} all the important information in
the system of equations, since the names of the variables have been
ignored, and the only connection with the variables is the location of
their coefficients in the matrix. It is important to realize that the
augmented matrix is just that, a matrix, and \emph{not} a system of
equations. In particular, the augmented matrix does not have any
``solutions,'' though it will be useful for finding solutions to the
system of equations that it is associated with. (Think about your
objects, and review {Proof Technique L}.) However, notice that an
augmented matrix always belongs to some system of equations, and vice
versa, so it is tempting to try and blur the distinction between the
two. Here is a quick example.

\protect\hyperlink{example-amaa-augmented-matrix-for-archetype-a}{\textbf{Augmented
matrix for Archetype A}}

\section{Row Operations}\label{row-operations}

An augmented matrix for a system of equations will save us the tedium of
continually writing down the names of the variables as we solve the
system. It will also release us from any dependence on the actual names
of the variables. We have seen how certain operations we can perform on
equations
(\href{http://linear.ups.edu/html/section-RREF.html}{Definition EO})
will preserve their solutions
(\href{http://linear.ups.edu/html/section-RREF.html}{Theorem EOPSS}).
The next two definitions and the following theorem carry over these
ideas to augmented matrices.

\protect\hyperlink{definition-ro-row-operations}{\textbf{Definition RO:
Row Operations}}. The following three operations will transform an
\$m\textbackslash{}times n\$ matrix into a different matrix of the same
size, and each is known as a \emph{row operation}.

\begin{enumerate}
\def\labelenumi{\arabic{enumi}.}
\tightlist
\item
  Swap the locations of two rows.
\item
  Multiply each entry of a single row by a nonzero quantity.
\item
  Multiply each entry of one row by some quantity, and add these values
  to the entries in the same columns of a second row. Leave the first
  row the same after this operation, but replace the second row by the
  new values.
\end{enumerate}

We will use a symbolic shorthand to describe these row operations:

\({-2}R_{1} \rightarrow R_{2}\)

\begin{enumerate}
\def\labelenumi{\arabic{enumi}.}
\tightlist
\item
  \(R_{i} \leftrightarrow R_{j}\): Swap the location of rows \$i\$ and
  \$j\$.
\item
  \(\alpha R_{i}\): Multiply row \(i\) by the nonzero scalar \(\alpha\).
\item
  \(\alpha R_{i} + R_{j}\): Multiply row \$i\$ by the scalar
  \$\textbackslash{}alpha\$ and add to row \$j\$.
\end{enumerate}

\protect\hyperlink{definition-rem-row-equivalent-matrices}{\textbf{Row-Equivalent
Matrices}}: Two matrices, \$A\$ and \$B\$, are \emph{row-equivalent} if
one can be obtained from the other by a sequence of row operations.

\protect\hyperlink{example-TREM}{\textbf{Example TREM: Two
row-equivalent matrices}}

The matrices

\(A=\begin{bmatrix} 2&-1&3&4\\ 5&2&-2&3\\ 1&1&0&6 \end{bmatrix}\)
\(B=\begin{bmatrix}  1&1&0&6\\  3&0&-2&-9\\  2&-1&3&4  \end{bmatrix}\)

are row-equivalent as can be seen from

\[\begin{bmatrix}
2&-1&3&4\\
5&2&-2&3\\
1&1&0&6
\end{bmatrix}\] \[R_{1} \rightarrow R_{3}\] \[\begin{bmatrix}
1&1&0&6\\
5&2&-2&3\\
2&-1&3&4
\end{bmatrix}\] \[{-2}R_{1} \rightarrow R_{2}\] \[\begin{bmatrix}
\&1&0&6\\
3&0&-2&-9\\
2&-1&3&4
\end{bmatrix}\]

We can also say that any pair of these three matrices are
row-equivalent.

Notice that each of the three row operations is reversible ({Exercise
RREF.T10}), so we do not have to be careful about the distinction
between ``\$A\$ is row-equivalent to \$B\$'' and ``\$B\$ is
row-equivalent to \$A\$.'' ({Exercise RREF.T11})

The preceding definitions are designed to make the following theorem
possible. It says that row-equivalent matrices represent systems of
linear equations that have identical solution sets.

\end{document}
